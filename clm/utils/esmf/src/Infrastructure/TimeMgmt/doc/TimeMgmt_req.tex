% $Id$

\section{Requirements}

\subsection{Date and Time Utility Requirements}

\begin{itemize}

\item Support \htmlref{no-leap}{glos:noleap} and Gregorian calendars.

\item Provide the following calculations:

\begin{itemize}

\item Given a date, compute a new date that is either earlier or later by a specified time interval.

\item Given two values of date compute the time interval between them.

\item Verify whether or not one date is later than another.

\item Compute the \htmlref{day of year}{glos:dayofyear} given date.

\end{itemize}

\item Handle a variety of representations of time intervals.  The initial requirement is supporting a 
precision of 1 second.  Future requirements include handling millisecond discretizations and floating 
point representations of time intervals.

\item Handle time intervals with a range of 20,000 years.
 
\item Date to string conversion.
\end{itemize}

\subsection{Time Manager Requirements}
\begin{itemize}

\item User will specify timestep size, start and stop dates. User may optionally specify the base date. 
By default the base date equals the start date.

\item Support changing the timestep size during a simulation. 

\item Provide functions to:

\begin{itemize}
\item Query timestep size, start and base dates.

\item Query current timestep number and properties of the current timestep such as date, 
time, and day of year at the endpoint of the current timestep.

\item Convert between time and date.
\end{itemize}

\item Provide alarm functions:

\begin{itemize}
\item Alarms that go off periodically can be specified by a period and an offset.

\item Alarms that go off when year and month boundaries are crossed.

\item A component or parameterization queries whether its alarm is on or off.
\end{itemize}

\item Must be able to operate in a "restart" mode.

\end{itemize}







